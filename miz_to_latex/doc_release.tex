\documentclass{article}
\usepackage[utf8]{inputenc}
\usepackage{fancyhdr}

\title{Miz To LaTeX - DCS World\\ \large Documentation du Script \texttt{run\_gui.sh}}
\author{Kerboul}
\date{\today}

% Définition de l'en-tête et du pied de page
\pagestyle{fancy}
\fancyhf{}
\rhead{\thepage}
\lfoot{Kerboul}
\rfoot{Page \thepage}

\begin{document}

\maketitle

\section{Introduction}
Cette documentation présente en détail l'utilisation et l'implémentation du script \texttt{run\_gui.sh}. Ce script est le point d'accès principal pour la génération de documents LaTeX à partir des briefings de missions DCS.

\section{Script \texttt{run\_gui.sh}}
Le script \texttt{run\_gui.sh} est le moyen principal pour lancer le programme de génération de documents LaTeX à partir des briefings DCS.

\subsection{Utilisation}
Pour exécuter le script, utilisez la commande suivante :
\begin{verbatim}
bash run_gui.sh
\end{verbatim}

\subsection{Fonctionnalités}
\begin{itemize}
    \item Lancement du script \texttt{run\_gui.py}.
    \item Interface graphique pour sélectionner le fichier de mission DCS.
    \item Extraction automatique des informations du briefing de mission.
    \item Création d'un document LaTeX bien formaté.
    \item Nettoyage automatique des fichiers temporaires.
\end{itemize}

\subsection{Dépendances}
Le script \texttt{run\_gui.sh} utilise les modules suivants :
\begin{itemize}
    \item \texttt{tkinter} : Pour l'interface graphique.
    \item \texttt{zipfile}, \texttt{os}, \texttt{re} : Pour le traitement des fichiers.
\end{itemize}

\section{Conclusion}
Ce script offre une solution efficace pour générer des documents LaTeX à partir des briefings DCS. Assurez-vous d'avoir les dépendances nécessaires et testez le script avant la distribution.

\end{document}

